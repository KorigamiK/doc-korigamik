\documentclass{korigamik}
\usepackage{lipsum}

\title{KorigamiK's Test\\ Document}
\author{KorigamiK}

\begin{document}

\maketitle

\tableofcontents

\section{Introduction}
This is an example of the \texttt{KorigamiK} document class.
Articles should always start with a section title, which will usually be `Introduction' or some such.

This document demonstrates the features of the document class.
Here are some logos and abbreviations:
\TeX, \pdfTeX, \BibTeX, \MF, \MP, \LaTeX, \LaTeXe,
\mbox{\ConTeXt}, \pdfLaTeX

On the following pages are shown the document divisions and list environments.

\section{Section headings}
The fact is that his precocity in vice was awful. At five months of age he
used to get into such passions that he was unable to articulate. At six
months, I caught him gnawing a pack of cards. For seven months he was in
the constant habit of catching and kissing the female babies. At eight
months he peremptorily refused to put his signature to the Temperance
pledge.
\paragraph{Text from Edgar Allen Poe's `Never bet the Devil your head'.}

\subsection{Subsection}
\lipsum[2]

\begin{figure}[htbp]
	\centering
	\square
	\caption{A square}
	\label{fig:square}
\end{figure}

\section{Conclusion}
\lipsum[3]

\section{Lists}
An itemized list:
\begin{itemize}
	\item one,
	\item two, and
	\item three.
	      A nested itemized list:
	      \begin{itemize}
		      \item one,
		      \item two, and
		      \item three.
	      \end{itemize}
\end{itemize}
An enumerated list:
\begin{enumerate}
	\item one,
	\item two, and
	\item three.
\end{enumerate}
A description list:
\begin{description}
	\item [one] the first,
	\item [two] the second, and
	\item [three] the third. A nested description list:
	      \begin{description}
		      \item [one] the first,
		      \item [two] the second, and
		      \item [three] the third.
	      \end{description}
\end{description}

\end{document}
